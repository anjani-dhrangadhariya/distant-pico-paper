\documentclass[10.7pt,]{article}

\usepackage[letterpaper, margin=2.54cm, top=2.54cm]{geometry}
\usepackage[super,comma,sort&compress]{natbib}
\usepackage{lmodern}
\usepackage{authblk} % To add affiliations to authors
\usepackage{amssymb,amsmath}
\usepackage{wrapfig}
\usepackage{graphicx,grffile}
\usepackage[labelfont=bf,labelsep=period]{caption}
\usepackage{ifxetex,ifluatex}
\usepackage{fixltx2e} % provides \textsubscript
\ifnum 0\ifxetex 1\fi\ifluatex 1\fi=0 % if pdftex
  \usepackage[T1]{fontenc}
  \usepackage[utf8]{inputenc}
\else % if luatex or xelatex
  \ifxetex
    \usepackage{mathspec}
  \else
    \usepackage{fontspec}
  \fi
  \defaultfontfeatures{Ligatures=TeX,Scale=MatchLowercase}
    \setmainfont[]{Arial Narrow}
    \setsansfont[]{Century Gothic}
    \setmonofont[Mapping=tex-ansi]{Consolas}
\fi
% use upquote if available, for straight quotes in verbatim environments
\IfFileExists{upquote.sty}{\usepackage{upquote}}{}
% use microtype if available
\IfFileExists{microtype.sty}{%
	\usepackage{microtype}
	\UseMicrotypeSet[protrusion]{basicmath} % disable protrusion for tt fonts
}{}

\usepackage{lipsum} % for dummy text only REMOVE

\newtheorem{exm}{Example}


%==============================
% Customization to make the output PDF 
% look similar to the MS Word version
%==============================
% To prevent hyphenation
\hyphenpenalty=10000
\exhyphenpenalty=10000

% To set the sections font size
\usepackage{sectsty}
\allsectionsfont{\fontsize{11}{11}\selectfont}
\sectionfont{\fontsize{14}{14}\selectfont}
\subsectionfont{\bfseries\fontsize{13}{13}\selectfont}
\subsubsectionfont{\bfseries\fontsize{11}{11}\selectfont}
%\subsubsectionfont{\normalfont}

% Spacing
\usepackage{setspace}


% No new line after subsubsection
\makeatletter
%\renewcommand\subsubsection{\@startsection{subsubsection}{3}{\z@}%
%	{-3.25ex\@plus -1ex \@minus -.2ex}%
%    {-1.5ex \@plus -.2ex}% Formerly 1.5ex \@plus .2ex
%    {\normalfont}}
%\makeatother

\makeatletter % Reference list option change
\renewcommand\@biblabel[1]{#1.} % from [1] to 1
\makeatother %

% To set the doc title font
\usepackage{etoolbox}
\makeatletter
\patchcmd{\@maketitle}{\LARGE}{\bfseries\fontsize{15}{16}\selectfont}{}{}
\makeatother

% No page numbering
\pagenumbering{gobble}

\makeatletter
\def\maxwidth{\ifdim\Gin@nat@width>\linewidth\linewidth\else\Gin@nat@width\fi}
\def\maxheight{\ifdim\Gin@nat@height>\textheight\textheight\else\Gin@nat@height\fi}
\makeatother

% Scale images if necessary, so that they will not overflow the page
% margins by default, and it is still possible to overwrite the defaults
% using explicit options in \includegraphics[width, height, ...]{}
\setkeys{Gin}{width=\maxwidth,height=\maxheight,keepaspectratio}
\setlength{\parindent}{0pt}
\setlength{\parskip}{6pt plus 2pt minus 1pt}
\setlength{\emergencystretch}{3em}  % prevent overfull lines
\providecommand{\tightlist}{%
  \setlength{\itemsep}{0pt}\setlength{\parskip}{0pt}}
\setcounter{secnumdepth}{0}
% Redefines (sub)paragraphs to behave more like sections
\ifx\paragraph\undefined\else
\let\oldparagraph\paragraph
\renewcommand{\paragraph}[1]{\oldparagraph{#1}\mbox{}}
\fi
\ifx\subparagraph\undefined\else
\let\oldsubparagraph\subparagraph
\renewcommand{\subparagraph}[1]{\oldsubparagraph{#1}\mbox{}}
\fi
%==============================
\usepackage{hyperref}
\hypersetup{
	unicode=true,
	pdftitle={My Cool Title Here},
	pdfauthor={Author One, Author Two, Author Three},
	pdfkeywords={keyword1, keyword2},
	pdfborder={0 0 0},
	breaklinks=true
}
\urlstyle{same}  % don't use monospace font for urls

% Keywords command
\providecommand{\keywords}[1]
{
  \small	
  \textbf{Key words---} #1
}
%==============================

% reduce space between title and begining of page
\title{\vspace{-2em} Not So Weak-PICO$+$: Leveraging weak supervision for Patient, Interventions, Outcomes and Study type extraction for systematic review automation}
\date{\vspace{-5ex}}
\author[ ] {
    % Authors
    \bf\fontsize{13}{14}\selectfont
    Anjani Dhrangadhariya,\textsuperscript{\rm 1, 2}
    Henning M\"uller \textsuperscript{\rm 1, 2}
}
\affil[1]{Institute of Business Information Systems, University of Applied Sciences Western Switzerland (HES-SO Valais-Wallis), Sierre, Switzerland}
\affil[2]{Department of Computer Science, University of Geneva (UNIGE), Geneva, Switzerland}
\affil[*]{Corresponding author: Anjani Dhrangadhariya, Institute of Business Information Systems, University of Applied Sciences Western Switzerland (HES-SO Valais-Wallis), Sierre, Switzerland; anjani.dhrangadhariya@hevs.ch}
%==============================
\begin{document}
\maketitle
\vspace{2em} %separation between the affiliations and abstract
%==============================
\doublespacing
%==============================
\section{ABSTRACT}
\label{abstract}
%==============================
% Note: Abstract is limited to 250 words
%
\textbf{Objective:}
PICO entity extraction is a vital but time-consuming task for conducting systematic reviews. 
Supervised machine learning methods can help fully automate PICO entity extraction, but a lack of large, diversified annotated corpora restricts innovation and adoption of automated PICO recognition systems.
The largest-available PICO entity corpus is manually annotated, which is too expensive for the most scientific community.
Moreover, even the well-trained annotators disagree on the exact spans or entities constituting PICO mentions, often requiring resource-intensive post-annotation corrections.
Our objective is to investigate a weak supervision approach devising freely-available ontologies and expert-generated rules for PICO entity extraction that does not rely on hand-labelled data.\\
\textbf{Materials and Methods:}
Ontologies - UMLS + more, dictionaries and expert-generated rules for designed for PICO recognition along with weakly supervised models.\\
\textbf{Results:}
Weak-PICO+, a method for weakly supervised PICO entity recognition using medical and non-medical ontologies and expert-generated rules.
Unlike manual annotation, the method does not use any hand-labelled data and is quickly adaptable and extensible to other entities.
We demonstrate this through extracting an additional Study Type entity making PICO go PICOS without manual annotation effort.\\
\textbf{Conclusion:}
Weak supervision using weak-PICO+ for PICO entity recognition is not only feasible but also outperforms full supervision.\\
%
%
%


\keywords{Randomized Controlled Trials, Weakly-Supervised Machine Learning, Information Extraction, Evidence-based Medicine}
%
\clearpage
%==============================
\section{QUESTIONS TO ADDRESS}\label{ques}
%==============================
%

\begin{enumerate}
    \item Is it important to have high F1 score for the individual labeling functions?
    \item Similar question: Will individual LF false negatives impact the performance?
    \item What would you call low coverage? What should be the ideal coverage of the labeling functions?
    \item What if higher coverage leads to higher False Positives?
    \item Labeling function weights for the functions with high F1 score (and not a good coverage necessarily)?
    \item How is overlap calculated? are only positive classes calculated in the overlap? > Coverage is calculated over 0 and 1 labels excluding the -1 labels.
    \item Is coverage correlated to F1 score?
    \item How is abstain treated while calculating F1 score?
    \item Check out Trove for how "0" labels are calculated?
    \item What annotation effort or how much annotation effort will be saved using distant-pico?
\end{enumerate}

Learning rate for label model train should be lower. LR like 1.0, 0.5, 0.1, 0.05, 0.01 did not work for me.

Problem of high precision and low recall. Could it solved by reducing the abstains?



%
%
%
%==============================
\section{INTRODUCTION}\label{introduction}
%==============================
%
\begin{itemize}
    \item PICO recognition - define, importance in SRs, challenges, slow process
    \item Extension - PICOS, PICOTS, PICO+
    \item Corpora creation and annotation - time consuming
    \item approach for span recognition but not entity recognition
    \item entities = semantic units vs. span = maximum information/description
    \item weak supervision approaches - democratize
\end{itemize}
%
%
%
%==============================
\section{MATERIALS AND METHODS}\label{methods}
%==============================
%
\begin{itemize}
    \item datasets - EBM-PICO training, EBM-PICO test gold, Physio gold
    \item Labeling function design - Positive LF, Negative LF
    \item Candidate generation - Experiments
    \item Model training - Training models using strong, weak and combination annotations. - Experiments
    \item Evaluation
\end{itemize}
%
%
%
%==============================
\subsection{Datasets}\label{data}
%==============================
%
\textbf{EBM-PICO.} The EBM-PICO dataset developed by Nye \textit{et al.} consists of 5000 PICO entity/span annotated documents.~\footnote{A single document consists of a title and an abstract.}
It comes pre-divided into a training set (n=4,933) annotated through crowd-sourcing and an expert annotated test set (n=191) for evaluation purposes.
The dataset has annotations at two levels.
Span level PICO annotation encompasses the maximum amount of PICO descriptions that often span an entire sentence.
Entity-level PICO annotation cover fine-grained PICO information at the entity level with PICO classes further divided into fine-grained subclasses.


\textbf{EBM-PICO corrected.} The fine-grained PICO annotations in EBM-PICO test set (n=191), however, have several errors that lead to faulty evaluation of machine learning methods.
These errors emanate from the annotation process of EBM-PICO corpus that follows the sequence of fist annotating only the span-level PICO information and then restrict the entity-level PICO annotation to these larger spans.
This leads to specifically missing out on the the repeated mentions of the fine-grained PICO information that is not covered in the longer spans. 
We conducted an error-focused corpus re-annotation for the EBM-PICO gold corpus.


\textbf{EBM-PICO corrected + Study Type.} We annotated an additional entity for study type only for this corpus to measure the effectiveness of weak supervision approaches.
Our study type entity is restricted to whether a study is a ``Randomized Controlled Trial'' or not.
This annotation exercise was to demonstrate the application of weak supervision for entities beyond PICO.

\textbf{Physio set.} A test set comprising 153 PICO entity/span annotated documents from Physiotherapy and Rehabilitation RCTs (Randomized Controlled Trials) was used as an additional benchmark to evaluate the generalization power of our approach for this sub-domain


\begin{table}[h!]
\begin{center}
\begin{tabular}{| c | c | c | c |} 
\hline
 & P & I/C & O \\ 
\hline
0 & No label & No label & No label \\ 
1 & Age & Surgical & Physical \\ 
2 & Sex & Physical & Pain \\
3 & Sample size & Drug & Mortality \\
4 & Condition & Educational & Side effect \\
5 &  & Psychological & Mental \\
6 &  & Other & Other \\
7 &  & Control &  \\
\hline
\end{tabular}
\caption{P (Participant), I (Intervention) and O (Outcome) represent the coarse-grained labels which are further divided into respective fine-grained labels.}
\label{table:coarsefineconcept}
\end{center}
\end{table}


%
%
%
%==============================
\subsection{Labeling functions}\label{lfs}
%==============================
%
UMLS concepts as labeling functions. It is difficult to classify different Semantic groups as positive signals for the PICOS classes. For example, the Participant span can include information about different participant characteristics: disease, disorder, ethnicity, language or geographical location, age, gender, sex,... 
%
%
%
%==============================
\subsection{EVALUATION}\label{eval}
%==============================
%
\begin{itemize}
    \item Classic metrics (Precision, Recall, F1).
    \item No evaluation for the Study type entity
    \item Students t-test for significance
\end{itemize}
%
%
%
%==============================
\section{RESULTS}\label{results}
%
Table demonstrating the scores mentioned above.
%
%
%
%==============================
\section{DISCUSSION}\label{discussion}
%==============================
%
\paragraph{Error correction: } Errors and the error correction to the EBM-PICO gold set explained here. 
Why were errors corrected?
Errors like this can cause faulty evaluation of the machine learning issues when one mention is annotated but another is not annotated.
We suggest the annotators to be medical students with knowledge about informatics like Bioinformaticians given that the ultimate application of such corpora is to develop machine learning algorithms that are in accordance to .
%
%
%
%==============================
\section{CONCLUSION}\label{conclusion}
%==============================
%
OxO
%
%
%
%==============================
\section{Acknowledgements}\label{acknowledgements}
%==============================
%
OxO
%
%
%
%==============================
\bibliographystyle{vancouver}
\bibliography{literature}
%==============================

\end{document}