\documentclass[10.7pt,]{article}

\usepackage[letterpaper, margin=2.54cm, top=2.54cm]{geometry}
\usepackage[super,comma,sort&compress]{natbib}
\usepackage{lmodern}
\usepackage{authblk} % To add affiliations to authors
\usepackage{amssymb,amsmath}
\usepackage{wrapfig}
\usepackage{graphicx,grffile}
\usepackage[labelfont=bf,labelsep=period]{caption}
\usepackage{ifxetex,ifluatex}
\usepackage{fixltx2e} % provides \textsubscript
\ifnum 0\ifxetex 1\fi\ifluatex 1\fi=0 % if pdftex
  \usepackage[T1]{fontenc}
  \usepackage[utf8]{inputenc}
\else % if luatex or xelatex
  \ifxetex
    \usepackage{mathspec}
  \else
    \usepackage{fontspec}
  \fi
  \defaultfontfeatures{Ligatures=TeX,Scale=MatchLowercase}
    \setmainfont[]{Arial Narrow}
    \setsansfont[]{Century Gothic}
    \setmonofont[Mapping=tex-ansi]{Consolas}
\fi
% use upquote if available, for straight quotes in verbatim environments
\IfFileExists{upquote.sty}{\usepackage{upquote}}{}
% use microtype if available
\IfFileExists{microtype.sty}{%
	\usepackage{microtype}
	\UseMicrotypeSet[protrusion]{basicmath} % disable protrusion for tt fonts
}{}

\usepackage{lipsum} % for dummy text only REMOVE

\newtheorem{exm}{Example}


%==============================
% Customization to make the output PDF 
% look similar to the MS Word version
%==============================
% To prevent hyphenation
\hyphenpenalty=10000
\exhyphenpenalty=10000

% To set the sections font size
\usepackage{sectsty}
\allsectionsfont{\fontsize{10}{10}\selectfont}
%\sectionfont{\fontsize{10}{10}\selectfont}
\subsectionfont{\bfseries\fontsize{10}{10}\selectfont}
\subsubsectionfont{\normalfont}

% No new line after subsubsection
\makeatletter
\renewcommand\subsubsection{\@startsection{subsubsection}{3}{\z@}%
	{-3.25ex\@plus -1ex \@minus -.2ex}%
    {-1.5ex \@plus -.2ex}% Formerly 1.5ex \@plus .2ex
    {\normalfont}}
\makeatother

\makeatletter % Reference list option change
\renewcommand\@biblabel[1]{#1.} % from [1] to 1
\makeatother %

% To set the doc title font
\usepackage{etoolbox}
\makeatletter
\patchcmd{\@maketitle}{\LARGE}{\bfseries\fontsize{15}{16}\selectfont}{}{}
\makeatother

% No page numbering
\pagenumbering{gobble}

\makeatletter
\def\maxwidth{\ifdim\Gin@nat@width>\linewidth\linewidth\else\Gin@nat@width\fi}
\def\maxheight{\ifdim\Gin@nat@height>\textheight\textheight\else\Gin@nat@height\fi}
\makeatother

% Scale images if necessary, so that they will not overflow the page
% margins by default, and it is still possible to overwrite the defaults
% using explicit options in \includegraphics[width, height, ...]{}
\setkeys{Gin}{width=\maxwidth,height=\maxheight,keepaspectratio}
\setlength{\parindent}{0pt}
\setlength{\parskip}{6pt plus 2pt minus 1pt}
\setlength{\emergencystretch}{3em}  % prevent overfull lines
\providecommand{\tightlist}{%
  \setlength{\itemsep}{0pt}\setlength{\parskip}{0pt}}
\setcounter{secnumdepth}{0}
% Redefines (sub)paragraphs to behave more like sections
\ifx\paragraph\undefined\else
\let\oldparagraph\paragraph
\renewcommand{\paragraph}[1]{\oldparagraph{#1}\mbox{}}
\fi
\ifx\subparagraph\undefined\else
\let\oldsubparagraph\subparagraph
\renewcommand{\subparagraph}[1]{\oldsubparagraph{#1}\mbox{}}
\fi
%==============================
\usepackage{hyperref}
\hypersetup{
	unicode=true,
	pdftitle={My Cool Title Here},
	pdfauthor={Author One, Author Two, Author Three},
	pdfkeywords={keyword1, keyword2},
	pdfborder={0 0 0},
	breaklinks=true
}
\urlstyle{same}  % don't use monospace font for urls

% Keywords command
\providecommand{\keywords}[1]
{
  \small	
  \textbf{Key words---} #1
}
%==============================

% reduce space between title and begining of page
\title{\vspace{-2em} DISTANT-PICO: A distantly annotated corpus of Patients, Interventions, Outcomes and Study type to Support Language Processing for Clinical Literature}
\date{\vspace{-5ex}}
\author[ ] {
    % Authors
    \bf\fontsize{13}{14}\selectfont
    Anjani Dhrangadhariya,\textsuperscript{\rm 1, 2}
    Henning M\"uller \textsuperscript{\rm 1, 2}
}
\affil[1]{Institute of Business Information Systems, University of Applied Sciences Western Switzerland (HES-SO Valais-Wallis), Sierre, Switzerland}
\affil[2]{Department of Computer Science, University of Geneva (UNIGE), Geneva, Switzerland}
\affil[*]{Corresponding author: Anjani Dhrangadhariya, Institute of Business Information Systems, University of Applied Sciences Western Switzerland (HES-SO Valais-Wallis), Sierre, Switzerland; anjani.dhrangadhariya@hevs.ch}
%==============================
\begin{document}
\maketitle
\vspace{2em} %separation between the affiliations and abstract
%==============================

%==============================
\section{ABSTRACT}\label{abstract}
%==============================
%
\textbf{Objective:}  OxO\\
\textbf{Materials and Methods:} OxO\\
\textbf{Results:} OxO\\
\textbf{Conclusion:} OxO\\
%
%
%


\keywords{Randomized Controlled Trial, Semi-Supervised Machine Learning, Systematic Review, Deep Learning}
%
\clearpage
%==============================
\section{INTRODUCTION}\label{introduction}
%==============================
%
OxO
%
%
%
%==============================
\subsection{Background and Significance}\label{background}
%==============================
%
OxO
%
%
%
%==============================
\section{OBJECTIVES}\label{objectives}
%==============================
We develop a weak annotation approach, DISTANT-PICOS to obtain annotate freely-available clinical trial records.
We review the machine learning models that we use to extract individual elements from trial reports, and we empirically evaluate the accuracy of the system components.
We make the weakly annotated corpus freely accessible via our prototype website, by bulk data download from the open science platform Zenodo.
%
%
%
%==============================
\section{MATERIALS AND METHODS}\label{methods}
%==============================
%
%
%
%==============================
\subsection{Overview}\label{overview}
%==============================
%
DISTANT-PICO is a distant supervision approach that uses several thousand clinical trial studies to generate a weakly annotated corpus of PICOS entities.
The approach is demonstrated in the Figure XYZ and Table XYZ.
First...
Then...
Lastly...
Each step is elaborated below.
%
%
%
%==============================
\subsection{Data}\label{data}
%==============================
%
Explain about clinical trial org and how data is organized in XML/JSON files in this database.
ClinicalTrials.gov (CTO hereafter) indexes 391,704 clinical trial studies (as of October 10, 2021) and is maintained by the U.S. National Library of Medicine (NLM).
Information about each study stored on CTO is entered and updated by the principal investigator of the corresponding trial.
This information includes title and summary of a clinical trial, disease conditions and interventions investigated in the trial, primary and secondary study outcomes, participants' eligibility criteria, \textit{etc.}

CTO GUI allows querying for clinical trials and displays each individual record as a combination of structured tabular and unstructured free-text (see Figure~\ref{fig:CTO_example}).
This vast amount of information can be downloaded in XML (Extensible Markup language) or JSON (JavaScript Object Notation) format.
CTO provides an API to query and retrieve an individual study, a list of studies and also a method for the complete database crawl~\footnote{\url{https://clinicaltrials.gov/api/}}.
In this work, we manipulated the JSON format and will keep to discussing the information content on CTO from the JSONs rather than the GUI.

Explain what are the fields for each of the PICOS criteria...\\
Referring to the study structure of clinical trial record on CTO~\footnote{\url{https://clinicaltrials.gov/api/info/study_structure?fmt=JSON}}.
Where are the P (condition, age, gender, sample size) information stored.\\
Where are the I/C information stored\\
Where is the Outcome information stored\\
Where is the Study type information store\\ (Why do we care about both randomized and non-randomized clinical trials?)
%
%
%
%==============================
\subsection{Candidate generation}\label{candgen}
%==============================

\begin{itemize}
    \item Explain the candidate generation process.
    \item Download or crawl the CTO dump (include the date of crawl)
    \item The conception of source and targets.
    \item The retrieval of source and targets.
    \item The expansion of source and targets.
    \item The preprocessing of source and targets.
    \item The direct matching and scoring of source and targets.
    \item The indirect matching of source and targets for ``Outcome'' mentions.
    \item Explain the process of combining labeling operators for PICOS entities.
\end{itemize}
%
%
%
%==============================
\subsection{Model training}\label{modtrain}
%==============================
Training models using strong, weak and combination annotations.
%
%
%
%==============================
\section{EVALUATION}\label{eval}
%==============================
%
\begin{itemize}
    \item Classic metrics (Precision, Recall, F1).
    \item No evaluation for the Study type entity
    \item Students t-test for significance
\end{itemize}
%
%
%
%==============================
\section{RESULTS}\label{results}
%
Table demonstrating the scores mentioned above.
%
%
%
%==============================
\section{DISCUSSION}\label{discussion}
%==============================
%
OxO
%
%
%
%==============================
\section{CONCLUSION}\label{conclusion}
%==============================
%
OxO
%
%
%
%==============================
\section{Acknowledgements}\label{acknowledgements}
%==============================
%
OxO
%
%
%
%==============================
\bibliographystyle{vancouver}
\bibliography{literature}
%==============================

\end{document}